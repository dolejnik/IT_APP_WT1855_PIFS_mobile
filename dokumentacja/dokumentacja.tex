\documentclass[a4paper,11pt]{article}

\usepackage[top=1.5in, bottom=1.5in, left=0.8in, right=0.8in]{geometry}
\usepackage[utf8]{inputenc}
\usepackage[T1]{fontenc}
\usepackage[english,polish]{babel}
\usepackage{indentfirst}
\usepackage{multirow}
\usepackage{graphicx}
\usepackage{subfig}
\usepackage[all]{nowidow} 
\usepackage{titletoc}
\usepackage{listings}
\usepackage{color}
\usepackage{relsize}

\clubpenalty10000
\widowpenalty10000

\linespread{1.3}
\setcounter{secnumdepth}{5}
\setcounter{tocdepth}{5}
\begin{document}
% definicje 
\makeatletter
\newcommand{\linia}{\noindent\rule{\linewidth}{0.4mm}}
\renewcommand{\maketitle}{
\begin{titlepage}
    \begin{center} \LARGE 
    \textbf{\textsc{
	    Politechnika Wrocławska \\
	    Wydział Elektroniki}}
    \end{center}
	\linia

	\vspace{3cm}
	\begin{center}
		\LARGE \textsc{\@title}
	\end{center}
	
	\vspace{1cm}
	\noindent
	\begin{minipage}[t]{0.5\textwidth}
		\begin{flushleft}
			\textit{\small Autorzy:}\\
			\normalsize \textsc{\@author} \par
		\end{flushleft}
	\end{minipage}
	\begin{minipage}[t]{0.5\textwidth}
		\begin{flushright}
			\textit{\small Praca wykonana pod przewodnictwem:}\\
			\normalsize \textsc{dr inż. Marek Woda} \par
		\end{flushright}
	\end{minipage}
    % \vspace*{4cm}

    \vfill
    \begin{center}
    Wrocław, \@date
    \end{center}
\end{titlepage}%
}
% definicje (end)



\author{Samir Senhadri 123456\\ Adam Szady 200890 \\ Mateusz Chudzik 200755 \\ Dawid Olejnik 200275 \\ Maciej Bożemój 200641 \\ Maciej Mościński 200893 }
\title{Zastosowania Informatyki w Gospodarce - projekt\\ \vspace{1.5cm} Przepływ informacji firmy serwisowej\\}
\definecolor{anti-flashwhite}{rgb}{0.95, 0.95, 0.96}
\lstset{language=C++,
	basicstyle=\ttfamily,
	keywordstyle=\color{blue}\ttfamily,
	stringstyle=\color{red}\ttfamily,
	commentstyle=\color{green}\ttfamily,
	morecomment=[l][\color{magenta}]{\#},
	showstringspaces=false,
	breaklines=true,
	backgroundcolor=\color{anti-flashwhite},
	tabsize=4
}




\newpage
\maketitle
\newpage

\dottedcontents{section}[1.5em]{\addvspace{1pc}\bfseries}{1.5em}{0.7pc}

\renewcommand\lstlistingname{Listing}
\renewcommand\lstlistlistingname{Spis listingów}

\tableofcontents

\newpage

\makeatother

\section{Wstęp}
Niniejszy dokument stanowi całościowe sprawozdanie z prac projektowych wykonanych w ciągu ostatniego semestru z zamiarem utworzenia wszechstronnego systemu wspomagającego zarządzanie zleceniami w firmie serwisowej.
\subsection{Geneza projektu}
Idea projektu narodziła się w dość naturalny i przypadkowy sposób krótko po zawiązaniu grupy projektowej. Jeden z członków grupy usiłował od dwóch miesięcy dokonać naprawy gwarancyjnej swojego telefonu komórkowego w autoryzowanym serwisie producenta - abstrahując nawet od faktu, iż za pierwszym razem telefon nie został naprawiony poprawnie i musiał zostać ponownie oddany do naprawy, to czas reparacji urządzenia wydaje się niewspółmiernie duży w stosunku do uszkodzenia telefonu (zepsute gniazdo ładowania). Co więcej serwis nie potrafił właściwie określić, na jakim etapie naprawy znajdują się obecnie, przez co klientowi nie pozostało nic innego jak zaprzestać telefonów i cierpliwie czekać. Przypuszczalnie sytuacja taka wynikała po części z faktu, że serwis nie stosował żadnego usystematyzowanego procesu dotyczącego zarówno samych napraw jak i komunikacji z klientem. Na kanwie złych doświadczeń z serwisem powstał pomysł realizacji systemu upraszczającego i ujednolicającego przepływ informacji pomiędzy serwisem a jego klientem z wykorzystaniem nowoczesnych technologii.
\subsection{Analiza stanu rynku}
nic nie zostało zrobione
\section{Zakres projektu}
Podstawową fazą projektu było obmyślenie wymaganych funkcjonalności systemu, z podziałem na podstawowe (muszą koniecznie zostać zrealizowane aby system działał poprawnie) i rozszerzone (powiększą możliwości lub komfort użytkowania systemu, ale nie są niezbędne do jego funkcjonowania). Następnym krokiem było stworzenie schematu bazy danych SQL, a następnie równoległe utworzenie komponentu serwerowego komunikującego się z bazą danych i aplikacji webowej i mobilnej wykorzystującej API wystawione przez część serwerową.
\subsection{Cel projektu}
Celem projektu było wytworzenie intuicyjnego systemu komunikacji pomiędzy firmą serwisującą urządzenia elektroniczne a jej klientem. System miał być dostępny zarówno z poziomu przeglądarki internetowej jak i telefonu komórkowego opartego o system operacyjny Android. Docelowy produkt powinien umożliwić łatwe dopasowanie do wymagań konkretnego serwisu urządzeń elektronicznych chcącego skorzystać z usług projektowanego produktu, aby usystematyzować proces napraw i poprawić wizerunek u klienta.
\subsection{Funkcjonalności podstawowe}
W projekcie wyróżnione zostały następujące funkcjonalności podstawowe:
\begin{itemize}
	\item LOREM
\end{itemize}
\subsection{Funkcjonalności rozszerzone}
W projekcie wyróżnione zostały następujące funkcjonalności rozszerzone:
\begin{itemize}
	\item IPSUM
\end{itemize}
\subsection{Ryzyka projektowe}
\section{Technologie zastosowane w projekcie}
\subsection{Aplikacja webowa}
\subsection{Aplikacja mobilna}
Aplikacja mobilna została stworzona na telefony z system operacyjnym Android (wymagana minimalna wersja X.Y.Z)
\section{Plan projektu}
\subsection{Kamienie milowe}
\subsection{Wykres Gantta}
\subsection{Rzeczywisty nakład pracy i koszty} 
\section{Implementacja i wdrożenie projektu}
\subsection{Implementacja aplikacji webowej}
\subsection{Implementacja aplikacji mobilnej}
\subsection{Instalacja projektu i wymagania sprzętowe}
\section{Podsumowanie}

\newpage
\listoffigures
\addcontentsline{toc}{section}{Spis rysunków} 
\newpage
\listoftables
\addcontentsline{toc}{section}{Spis tablic}
\newpage
\lstlistoflistings
\addcontentsline{toc}{section}{Spis listingów}


\newpage
\addcontentsline{toc}{section}{Literatura}
\begin{thebibliography}{9}
\bibitem{cormen} Cormen T., Leiseron C., Rivest R., Wprowadzenie do algorytmów, WNT, 2001. 
\bibitem{kombi} Błażewicz J., Problemy optymalizacji kombinatorycznej, PWN, Warszawa, 1996.
\bibitem{tabu} Strona internetowa: http://www.cs.put.poznan.pl/wkotlowski/teaching/8-ts.pdf, 25.01.2013r.
\end{thebibliography}

\end{document}